\documentclass[12pt,t]{beamer}

\usepackage{graphicx}
\usepackage{minted}
\usepackage{amsmath}
\setbeameroption{hide notes}
\setbeamertemplate{note page}[plain]

% get rid of junk
\usetheme{default}
\beamertemplatenavigationsymbolsempty
\hypersetup{pdfpagemode=UseNone} % don't show bookmarks on initial view

% font
\usepackage{fontspec}
\setsansfont{TeX Gyre Heros}
\setbeamerfont{note page}{family*=pplx,size=\footnotesize} % Palatino for notes
% "TeX Gyre Heros can be used as a replacement for Helvetica"
% In Unix, unzip the following into ~/.fonts
% In Mac, unzip it, double-click the .otf files, and install using "FontBook"
%   http://www.gust.org.pl/projects/e-foundry/tex-gyre/heros/qhv2.004otf.zip

% named colors
\definecolor{offwhite}{RGB}{249,242,215}
\definecolor{foreground}{RGB}{255,255,255}
\definecolor{background}{RGB}{24,24,24}
\definecolor{title}{RGB}{214,174,107}
\definecolor{gray}{RGB}{155,155,155}
\definecolor{subtitle}{RGB}{152,255,104}
\definecolor{hilight}{RGB}{102,255,204}
\definecolor{vhilight}{RGB}{255,111,207}
\definecolor{lolight}{RGB}{155,155,155}
%\definecolor{green}{RGB}{125,250,125}

% use those colors
\setbeamercolor{titlelike}{fg=title}
\setbeamercolor{subtitle}{fg=subtitle}
\setbeamercolor{institute}{fg=gray}
\setbeamercolor{normal text}{fg=foreground,bg=background}
\setbeamercolor{item}{fg=foreground} % color of bullets
\setbeamercolor{subitem}{fg=gray}
\setbeamercolor{itemize/enumerate subbody}{fg=gray}
\setbeamertemplate{itemize subitem}{{\textendash}}
\setbeamerfont{itemize/enumerate subbody}{size=\footnotesize}
\setbeamerfont{itemize/enumerate subitem}{size=\footnotesize}

% page number
\setbeamertemplate{footline}{%
    \raisebox{5pt}{\makebox[\paperwidth]{\hfill\makebox[20pt]{\color{gray}
          \scriptsize\insertframenumber}}}\hspace*{5pt}}

% add a bit of space at the top of the notes page
\addtobeamertemplate{note page}{\setlength{\parskip}{12pt}}

% a few macros
\newcommand{\bi}{\begin{itemize}}
\newcommand{\ei}{\end{itemize}}
\newcommand{\ig}{\includegraphics}
\newcommand{\subt}[1]{{\footnotesize \color{subtitle} {#1}}}
\newcommand{\codepy}{\mintinline{python}}
\newcommand{\codehs}{\mintinline{haskell}}
\newcommand{\vitem}{\vfill\item}

\newcommand{\filler}[1]{%
        \begin{frame}
            \vfill
            \centering{\color{title}\large#1}
        \end{frame}
    }

% title info
\title{DuckTest}
\subtitle{Static Duck Type Checking for Python}
\author{Josh Rahm}
\institute{{Department of Computer Science} \\[2pt] {University of Colorado at Boulder}}
\date{\tt \scriptsize \today
\\
\href{https://github.com/jrahm/}{\tt \scriptsize github.com/jrahm}
}


\begin{document}

\setbeamercolor{item}{fg=title}
\usemintedstyle{monokai}
% title slide
{%
\setbeamertemplate{footline}{} % no page number here
\begin{frame}
  \titlepage
  \vfill
  \centering\tiny{(Template unbashedly modified from \href{https://github.com/kbroman/Talk_OpenAccess}{Karl Broman)}}
\end{frame} }

\filler{Motivation}

\begin{frame}{Problem}
    \vfill
    \vspace{-0.5cm}
    Python is way too susceptible to TypeErrors. The number of
    times I've gotten ``None type is not iterable'' or crashed
    because an \codepy{str} cannot be concated with \codepy{int},
    my favorite, has no attribute \codepy{contians},
    is just unbearable and extremely frustrating!
    \begin{description}
        \vitem[REPL] = {\color{title}{R}}un, try to {\color{title}{E}}val and {\color{title}{P}}uke {\color{title}{L}}oop
    \end{description}
    \vspace{0.5cm}

    It slows production to such a crawl, C++ would be better!
\end{frame}

\begin{frame}{Enter DuckTest}
    \subt{Powerful incecticide}
    \begin{itemize}
        \vitem DuckTest is yet another lint checker for Python, but with a different
        philosophy than the others.

        \vitem Instead of focusing on convention violations in the hope that better
        conventions will lead to bug-free code, DuckTest goes right for the
        throat and hunts down type errors before they crash your program!

        \vitem DuckTest is also written in Haskell; it is just so much better
        suited for this type of stuff than Python itself is.

    \end{itemize}
\end{frame}

\begin{frame}{Example}
    \subt{Code}
    \vspace{0.5cm}

    \begin{minipage}[t]{0.5\textwidth}
    \small\inputminted{python}{code/duck.py}
    \end{minipage}
    \begin{minipage}[t]{0.4\textwidth}
    \small\inputminted{python}{code/person.py}
    \end{minipage}
\end{frame}

\begin{frame}{Example (Cont.)}
    \small\inputminted{python}{code/body.py}
\end{frame}

\begin{frame}{Example (Cont.)}
    \subt{What's the Problem?}
    \begin{itemize}
        \vfill\item \codepy{test_feathers(person)} is guaranteed to fail. It
            is a blantant, easy to spot type-error for a human.
        \vfill\item \codepy{ruffle} requires its argument to have an attribute
            called \codepy{feathers}, but \codepy{Person} does not
        \vfill\item Easy for a human to spot, it's just a couple calls down!
        \vfill\item Yet no lint checkers found it! (They were too busy complaining
            about my lack of docstrings)
        \vfill\item DuckTest can find this in its sleep!
    \end{itemize}
\end{frame}

\begin{frame}{Can't we all just get along?}
    \subt{Duck-Typing $\not=$ Dynamic Typing!}

    Many static type systems have elements of duck typing. Consider
    the following C++:

    \vfill
    \inputminted{cpp}{code/cpp.cpp}
    \vfill

    If g++ can check that code at compile (actually link) time, why
    can't Python?
\end{frame}

\begin{frame}{What's the Catch?}
    \subt{Type-Erasure Semantics}
    \bi
    \vitem For DuckTest to work well, the code must be written using type-erasure
    semantics. That is, the program should not branch based on the type of
    an object at runtime. So, essentially no calls to \codepy{hasattr} or
    branching based on \codepy{__class__}
    \vitem Arguably, good code shouldn't do this anyway \ldots
    \vitem Not a horrible constraint, most programmers keep track of types by
    writing in this style anyway.
    \ei
\end{frame}

\filler{Implementation}

\begin{frame}{The Algorithm(s)}
    \begin{description}
        \vitem[Type Inference by Evaluation] -- Infering the type of some
        variable based on the type of the expression it is assigned to. This is
        used in conjunction with assignment to infer the type of a variable at assignment.

        \vitem[Type Inference by Observance] -- Infering the type of some
        variable by observing how it is used. This is for infering the types of the
        parameters to a function.

        \vitem[Type Matching] -- Given a two types $t_1$ and $t_2$ can $t_1$ act like $t_2$?
        This can be thought of as, ``is $t_2$ `smaller' than $t_1$.'' This is where the
        duck testing happens.
    \end{description}
    Because Python is so dynamic, these all happen at the same time!
\end{frame}

\begin{frame}{Modeling Python's Types}
    \vspace{0.5cm}
    {\large \color{subtitle} {Structural Typing}}

    \begin{quote}
        A structural type system (or property-based
        type system) is a major class of type system, in which type
        compatibility and equivalence are determined by the type's actual
        structure or definition, and not by other characteristics such as its
        name or place of declaration (Wikipedia)
    \end{quote}

    \vfill
    With some modifications \ldots
\end{frame}

\begin{frame}{The Model}
    \inputminted{haskell}{code/model.hs}
\end{frame}

\begin{frame}{Scalar Type}
    \vfill
    \codehs{Scalar (Maybe String) (Map String PyType)}
    \vfill
    The scalar type is the base structural type. It
    has an array of attributes and their types. It also
    contains an optional type name. While strictly speaking,
    the type name is not an actual part of a structural type,
    it is useful as we'll see.
    \vfill
\end{frame}

\begin{frame}{Functional Type}
    \vfill
    \codehs{Functional [PyType] PyType}
    \vfill
    A type representing a function. It simply has a list
    of possible arguments and a return type. It can be thought
    of as a scalar with a single method (in Python) `\_\_call\_\_',
    we still need it though, because without it, what would the type
    of `\_\_call\_\_' be?
    \vfill
\end{frame}

\begin{frame}{Alpha Type}
    \vfill
    \codehs{Alpha String PyType}
    \vfill
    Probably the most confusing type. It contains a name as well
    as a refrence to a type. It is used for recursive types like
    a LinkedList. Without this type, matching the type of a linked
    list with another would result in non-termination. The alpha
    type allows just the names to be compared and nothing more.
    \vfill
\end{frame}

\begin{frame}{Any Type}
    \vfill
    \codehs{Any}
    \vfill
    Conceptually this is the infinite Scalar. It contains all
    possible attributtes all of which have the Any type. The
    Any type is compatible with all other types. Conceptually,
    it is redundent since it is just a scalar with everything,
    but practically we cannot implement that efficiently.
    \vfill
\end{frame}

\begin{frame}{Void Type}
    \vfill
    \codehs{Void}
    \vfill
    Conceptually this is the empty Scalar. Strictly speaking, it
    is redundant, but for debugging and ease of use it is included. This
    is considered the `zero' type (yes, types can be monoids!),
    Making it the opposite of the Any type. It is incompatible with everything.
    \vfill
\end{frame}

\begin{frame}{Back to Eighth Grade}
    \subt{An Algebra for PyType}

    \begin{description}
        \vitem[Union $t_1$ $t_2$] -- Returns a type that can act as either $t_1$ or $t_2$
        \vitem[Intersection $t_1$ $t_2$] -- Returns a type that $t_1$ and $t_2$ can act like
        \vitem[Difference $t_1$ $t_2$] -- Returns what is needed to allow $t_2$ to act like $t_1$
        \vitem[Singleton $a$ $t$] -- Returns a structural type with exactly one attribute $a::t$
    \end{description}
\end{frame}

\begin{frame}{Union}
    \subt{Some Rules}
    \[
      \begin{array}{ll}
          & Union\ Any\ \_ = Any \\
          & Union\ \_\ Any = Any \\
          & Union\ Void\ t_1 = t_1 \\
          & Union\ t_1 Void = t_1 \\
          & Union\ (Scalar\ m_1)\ (Scalar\ m_2) = Scalar\ (m_1\cup_r m_2) \\
          & Union\ scalar\ fn = Union\ scalar\ (singleton\ \_\_call\_\_\ fn) \\
          & \vdots
    \end{array}
    \]

    \vfill
    \centering\tiny{($\cup_r$ is the recursive union operator -- unionWith Union)}
\end{frame}

\begin{frame}{Intersection}
    \subt{Some Rules}
    \[
      \begin{array}{ll}
          & Inter\ Any\ t_1 = t_1 \\
          & Inter\ t_1\ Any = t_1 \\
          & Inter\ Void\ t_1 = Void \\
          & Inter\ t_1 Void = Void \\
          & Inter\ (Scalar\ m_1)\ (Scalar\ m_2) = Scalar\ (m_1\cap_r m_2) \\
          & Inter\ scalar\ fn = Inter\ scalar\ (singleton\ \_\_call\_\_\ fn) \\
          & \vdots
    \end{array}
    \]

    \vfill
    \centering\tiny{($\cap_r$ is the recursive intersection operator -- intercectWith Inter)}
\end{frame}

\begin{frame}{Difference}
    \subt{Some Rules}
    \[
      \begin{array}{ll}
          & Diff\ Any\ \_ = Any \\
          & Diff\ \_\ Any = Void \\
          & Diff\ Void\ t_1 = Void \\
          & Diff\ t_1 Void = t_1 \\
          & Diff\ (Scalar\ m_1)\ (Scalar\ m_2) = Scalar\ (m_1\setminus_r m_2) \\
          & Diff\ scalar\ fn = Diff\ scalar\ (singleton\ \_\_call\_\_\ fn) \\
          & \vdots
    \end{array}
    \]

    \vfill
    \centering\tiny{($\setminus_r$ is the recursive difference operator -- differenceWith Diff)}
\end{frame}

\begin{frame}{Using Our Algebra}
    \subt{Function Argument Inference}

    \bi
        \vitem A function is a set of arguments and statements and statements
        have expressions.

        \vitem To infer the type of an argument, we use the observant type inference.
        \bi
            \vitem Each time we observe an attribute $a :: t_a$ access on argument $v$
            we say $v :: t \Rightarrow v :: (t \cup singleton\ a\ t_a)$
            \vitem More formal
            \[
                \frac {
                    \gamma[v] = t\qquad v.a :: t_a
                } {
                    v.a \rightarrow \gamma[v/t \cup singleton\ a\ t_a]
                }
            \]
            \vitem Essentially, we walk through the expression and union together
            all observed attribute accesses.
        \ei
    \ei
\end{frame}

\begin{frame}{Using Our Algebra}
    \subt{Function Return Type Inference}

    \bi
        \vitem We use evaluation type inference for this.
        \bi
            \vitem Find all return statements $return\ expr$
            \vitem Infer the type of $expr :: t$
            \vitem The function is said to return the \emph{intersection}
                of all return expression types.
        \ei
    \ei
\end{frame}

\begin{frame}{Using Our Algebra}
    \subt{Type Matching}

    \bi
        \vitem Function calls have expected types for parameters.
        \vitem Match each argument expression with the expected type.
        \vitem $match\ t_h\ t_e = (diff\ t_e\ t_h = Void)$

        If the type expected is less than or equal to type the type we have,
        the two are compatible.
    \ei
\end{frame}

\begin{frame}
    \vfill
    \centering\large Demo
\end{frame}



\end{document}
