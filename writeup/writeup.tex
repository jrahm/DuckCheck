\documentclass{scrartcl}

\usepackage{graphicx}
\usepackage{minted}
\usepackage{color}
\usepackage{fancyhdr}
\usepackage[margin=1.0in]{geometry}
\usepackage{multicol}
\usepackage{titlesec}
\usepackage{filecontents}
\usepackage{url}

\definecolor{secblue}{rgb}{0.18, 0.27, 0.33}

\usepackage{xhfill}
\usepackage{breqn}


\newcommand{\backslashx}[0]{\textsc{Joshua Rahm}}

\titleformat{\section}
  {\Large\titlefont\color{secblue}}{\thesection}{1em}{}

\titleformat{\subsection}
  {\large\titlefont\color{secblue}}{\thesection}{1em}{}

\titleformat{\subsubsection}
{\normalsize\titlefont\color{gray}}{\thesection}{1em}{}

\definecolor{bar}{rgb}{0.90, 0.60, 0.30}
\definecolor{bardark}{rgb}{0.45, 0.30, 0.15}
\definecolor{darkgreen}{rgb}{0.0, 0.30, 0.0}
\fancypagestyle{plain}{%
    \fancyhf{}
    % \fancyhead[R]{\textcolor{blue}{\hrule}}
    \fancyhead[L]{\textcolor{darkgray}{\textbf{\backslashx}}}
    \fancyhead[R]{\textcolor{darkgray}{\textbf{DuckTest: Static Type Checking for Python}}}
    \renewcommand{\headrulewidth}{2pt}% 2pt header rule
    \renewcommand{\headrule}{\hbox to\headwidth{%
        \color{bar}\leaders\hrule height \headrulewidth\hfill}}

    % \fancyhead[R]{\textcolor{blue}{\hrule}}
    \fancyfoot[R]{\thepage}
    \fancyfoot[L]{\today}
    \renewcommand{\footrulewidth}{2pt}% 2pt header rule
    \renewcommand{\footrule}{\hbox to\headwidth{%
        \color{bar}\leaders\hrule height \headrulewidth\hfill}}
}

\newcommand{\hastype}{{\ ::\ }}

\newcommand{\maths}[1]{%
    \vspace{0.25cm}
    \begin{dmath*}
    #1
    \end{dmath*}
    \vspace{0.25cm}
}

\usepackage[framemethod=TikZ]{mdframed}% Required for creating boxes
\mdfdefinestyle{left}{
    linecolor=gray, % Outer line color
%    frametitlerule=true, % Title rule - true or false
%    frametitlerulecolor=gray, % Title rule color
%    frametitlerulewidth=0.5pt % Title rule width
%    outerlinewidth=0.5pt, % Outer line width
%    roundcorner=0pt, % Amount of corner rounding
%    innertopmargin=0pt, % Top margin
%    innerbottommargin=0pt, % Bottom margin
%    innerrightmargin=0pt, % Right margin
%    innerleftmargin=10pt, % Left margin
%    leftmargin=12pt, % Left margin
%    backgroundcolor=white, % Box background color
%    frametitlebackgroundcolor=white, % Title background color
%    frametitlefont=\Large, % Title heading font specification
%    font=\small,
    leftline=false,
    topline=false,
    rightline=false,
    bottomline=false
}

\mdfsetup{nobreak=true}

\newcommand{\inputcode}[1]{
\begin{mdframed}[style=left]
\noindent\makebox[\textwidth][c]{%
    \begin{minipage}{0.8\textwidth}
        \inputminted[linenos]{python}{#1}
    \end{minipage}
}
\end{mdframed}
}


\pagestyle{plain}

\title{DuckTest}
\subtitle{{\textcolor{bardark}{A Brief Tour of Static Duck Type Checking for Python}}}
\author{Joshua Rahm \\
\footnotesize Department of Computer Science \\[2pt]
\footnotesize University of Colorado}
\date{}


\begin{filecontents}{bib.bib}
@misc{logic,
  author = {Ravi Chugh, Patrick M. Rondon, Ranjit Jhal},
  title = {{Nested Refinements: A Logic for Duck Typing}},
  howpublished = "\url{http://people.cs.uchicago.edu/~rchugh/static/papers/popl12-nested.pdf}",
  note = "[Online; accessed 10-November-2015]"
}

@misc{nst,
  title = {{Nomitive and Structural Typing}},
  howpublished = "\url{http://c2.com/cgi/wiki?NominativeAndStructuralTyping}",
  note = "[Online; accessed 13-November-2015]"
}

@misc{dynamic,
  author = {Jonathan Eifrig, Scott Smith, Valery Trifonov},
  title = {{Type Inference for Recursively Constrained Types and its Application to OOP}},
  howpublished = "\url{http://citeseerx.ist.psu.edu/viewdoc/download;jsessionid=F4F3BAF0FC6AB2A0681AA4428CFD4C4A?doi=10.1.1.92.6519&rep=rep1&type=pdf}",
  note = "[Online; accessed 13-November-2015]"
}
\end{filecontents}


\begin{document}
\nocite{*}
\maketitle
\hrule

% \raggedcolumns
\begin{multicols}{2}
% \raggedcolumns

\begin{abstract}
    Dynamic typing is a very hot field right now. It decreases the amount
    of work needed to complete simple tasks. Without doubt, dynamic languages
    have flexibility that is unmatched by any statically typed language no
    matter how powerful the type system. This flexibility comes at a cost, however.
    Namely in static analysis, particularly type verifiers. The lack of a static
    type system can lead to very onerous errors at runtime. This is where DuckTest
    comes in. DuckTest is a Python static analyzer that uses Data Flow Analysis
    and Type inference to detect TypeErrors statically while retaining as
    much flexibility as possible.
\end{abstract}

\section*{Introduction}

The duck test refers to the ever so popular statement ``If it looks like a duck,
swims like a duck, and quacks like a duck, then it probably is a duck.'' While
no sound logic for formal proofs, this type of logic has found its way into some
of the most popular programming languages today. These ``Duck'' type systems allow
an object $o_1$ to act as some type $t_1$ if $o_1$ implements all the features of
that type regardless of what the named type of $o_1$ actually is.

This type system makes is very flexible and allows the programmer to use
different types under the same context. Contrary to what many programmers
think, a duck type system is not synonomous with a \emph{dynamic} type
system, but the two are often together.

DuckTest is an application employing techniques such as Data Flow Analysis and
Type Inference to statically screen Python source code for potential type errors.
Currently DuckTest only supports a relatively limited subset of Python, but
extensions and improvements will continue to be added.

\section*{Motivation}

Python and other dynamically typed languages are very useful tools to solve a wide
range of different problems. While these languages are very flexible and very quick
to prototype in, their very dynamic style makes it very difficult for static analyzers
to properly check the source for what one would consider obvious errors.

\section*{Proposal}

I propose that as an implementation project, I will implement a subset of the
static checker described in the previous project for a subset of the Python
programming language. The exact subset has yet to be defined, but it will
operate under sanely typed Python programs, meaning programs which are intended
to be type-safe and adhere to type-erasure semantics.

The more exact terms of the implementation are described in later sections.

The project name is Hiss, after (of course) the sound a snake makes.

\section*{Duck Typing}

Duck typing, specifically the ability to infer types in a Duck-Typed system,
is at the heart of the implementation of Hiss.

A misconception is that Duck Typing is only associated with dynamically typed
languages. This is false, although it is true that most modern dynamically typed
languages implement a form of duck typing, it is not rare to see a statically
typed language implement some duck typing. A very common example is the following
example using C++ templates.

%\inputminted{cpp}{test1.cpp}

This code snippet will compile, and will behave as expected. C++ is able
to infer types properly, statically check them, and compile. Even though
\emph{Person} and \emph{Duck} are not a part of the same inheritance structure,
they both may still be used in the \emph{InTheForest} function because they
happen to have the correct functions.

If C++ can compile this, then the python analog should be checkable:

%\inputminted{python}{test1.py}

Alas, there are very few, if any, static verifiers out there that can actually
check this code. In fact, changing the definition of \emph{InTheForest}
to be:

\begin{minted}{python}
def InTheForest(duck):
    duck.quack()
    duck.walk()
    duck.feathers()
\end{minted}

there is absolutely no indication from pylint (a common python lint checker) that
this will blatantly fail at runtime because a \emph{Person} does not have the
method \emph{feathers}. Yet a similar change in the C++ code will result in a
very quick compile error.

\subsection*{Duck Type Inference \& Structural Typing}

As stated previously, at the heart of Hiss is the need to be able to use logic
to infer the types of a Duck-Typed programming language. This requires a very
different approach to typing than traditional type systems like those in Java
or C++. What is required is a model of types which is not based on the name of
the type, but rather a model of typing based on the attributes of the values.
This a concept that Structural typing attempts to formalize. In a structural
type system, for an object $O$ to be compatible with a type $T$, $O$ must implement
a superset of attributes defined in $T$.\cite{nst}

\subsection*{Constrained Typing with Type Inference}

More languages these days like Haskell are implementing a typing discipline
knows as constrained types. These types allow the programmer to define function
and attributes to a \emph{class of data types}. For example, a ``Readable'' type
may have a function called \emph{read}, likewise a ``Monoid'' type may have a
function called ``zero'' and a function called ``append''. Readable and Monoid are
example of constraints on a type. When performing type-inference on them, they
help to provide the most generic, but complete constraint for a type. So if a
type $\alpha$ is observed to be used in an ``append'' call, it can be inferred
that $\alpha$ is \emph{constrained} by the class Monoid.

\subsection*{Inference}

When witnessing an assignment to a variable $v$, we can extrapolate the
expected \emph{structural type} of $v$. From there, as Hiss linearly traces
through the program, every method invoked on $v$ and every attribute accessed
on $v$ can be verified to be a part of $v$'s structural type. If not, a warning
will be raised to notify the programmer that a potential type error has
occurred.

In the case that an assignment to $v$ is a base case, meaning the assignment
is from either a string literal, a number, or a constructor, then the full
type as well as the structural type is then known by Hiss and further inferences
may be made.

\subsubsection*{Functions}

Structural types may be given to a function's return type and argument types by
this inference of the structural types of the arguments as well as the return
type. This can be used both to infer types of the parameters and the return type
so that all invocations of the function may also be inferred with the correct
structural types.

\subsubsection*{Any Type}

To completely express this type system, there must be a notion of the "any type"
meaning that there are no clues to the actual type of a variable and therefore
its structural type is a wild card of sorts and cannot be determined. The Any Type
will come into play when using an opaque library and it is not possible to determine

\section*{Timeline}

As the weeks progress, I will work on the implementation of this project. The
implementation will be written in Haskell since there is a fairly mature and reliable
Python parser for the language. After the first week, I plan to have basic scaffolding
done, then over the break and the following week I plan to implement some of the
basic functionality. Finally, the last week of the semester will be dedicated into
formalizing the results and putting together a short writeup about the findings.

\bibliographystyle{abbrv}
\bibliography{bib}

\vspace{\textheight}
\end{multicols}
\end{document}
