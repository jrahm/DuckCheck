\documentclass{scrartcl}

\usepackage{graphicx}
\usepackage{minted}
\usepackage{latexsym}
\usepackage{color}
\usepackage{fancyhdr}
\usepackage[margin=1.0in]{geometry}
\usepackage{multicol}
\usepackage{titlesec}
\usepackage{filecontents}
\usepackage{url}
\usepackage[hidelinks]{hyperref}

\usemintedstyle{autumn}

\definecolor{secblue}{rgb}{0.18, 0.27, 0.33}

\newcommand*{\fvtextcolor}[2]{\textcolor{#1}{#2}}

\usepackage{xhfill}
\usepackage{breqn}


\newcommand{\backslashx}[0]{\textsc{Joshua Rahm}}

\titleformat{\section}
  {\Large\titlefont\color{secblue}}{\thesection}{1em}{}

\titleformat{\subsection}
  {\large\titlefont\color{secblue}}{\thesection}{1em}{}

\titleformat{\subsubsection}
{\normalsize\titlefont\color{gray}}{\thesection}{1em}{}

\definecolor{bar}{rgb}{0.90, 0.60, 0.30}
\definecolor{bardark}{rgb}{0.45, 0.30, 0.15}
\definecolor{darkgreen}{rgb}{0.0, 0.30, 0.0}
\fancypagestyle{plain}{%
    \fancyhf{}
    % \fancyhead[R]{\textcolor{blue}{\hrule}}
    \fancyhead[L]{\textcolor{darkgray}{\textbf{\backslashx}}}
    \fancyhead[R]{\textcolor{darkgray}{\textbf{DuckTest: Static Type Checking for Python}}}
    \renewcommand{\headrulewidth}{2pt}% 2pt header rule
    \renewcommand{\headrule}{\hbox to\headwidth{%
        \color{bar}\leaders\hrule height \headrulewidth\hfill}}

    % \fancyhead[R]{\textcolor{blue}{\hrule}}
    \fancyfoot[R]{\thepage}
    \fancyfoot[L]{\today}
    \renewcommand{\footrulewidth}{2pt}% 2pt header rule
    \renewcommand{\footrule}{\hbox to\headwidth{%
        \color{bar}\leaders\hrule height \headrulewidth\hfill}}
}

\newcommand{\hastype}{{\ ::\ }}

\newcommand{\maths}[1]{%
    \vspace{0.25cm}
    \begin{dmath*}
    #1
    \end{dmath*}
    \vspace{0.25cm}
}

\usepackage[framemethod=TikZ]{mdframed}% Required for creating boxes
\mdfdefinestyle{left}{
    linecolor=gray, % Outer line color
%    frametitlerule=true, % Title rule - true or false
%    frametitlerulecolor=gray, % Title rule color
%    frametitlerulewidth=0.5pt % Title rule width
%    outerlinewidth=0.5pt, % Outer line width
%    roundcorner=0pt, % Amount of corner rounding
%    innertopmargin=0pt, % Top margin
%    innerbottommargin=0pt, % Bottom margin
%    innerrightmargin=0pt, % Right margin
%    innerleftmargin=10pt, % Left margin
%    leftmargin=12pt, % Left margin
%    backgroundcolor=white, % Box background color
%    frametitlebackgroundcolor=white, % Title background color
%    frametitlefont=\Large, % Title heading font specification
%    font=\small,
    leftline=false,
    topline=false,
    rightline=false,
    bottomline=false
}

\mdfsetup{nobreak=true}

\newcommand{\inputcode}[1]{
\begin{mdframed}[style=left]
\noindent\makebox[\textwidth][c]{%
    \begin{minipage}{0.8\textwidth}
        \inputminted[linenos]{python}{#1}
    \end{minipage}
}
\end{mdframed}
}


\pagestyle{plain}

\title{DuckTest}
\subtitle{{\textcolor{bardark}{Type Checking Duck-Typed Python Source}}}
\author{\href{http://jarahm.com/}{Joshua Rahm} \\
\footnotesize Department of Computer Science \\
\footnotesize University of Colorado \\
\footnotesize \href{https://github.com/jrahm/DuckTest}{github.com/jrahm/DuckTest}
}
\date{}


\begin{filecontents}{bib.bib}
@misc{logic,
  author = {Ravi Chugh, Patrick M. Rondon, Ranjit Jhal},
  title = {{Nested Refinements: A Logic for Duck Typing}},
  howpublished = "\url{http://people.cs.uchicago.edu/~rchugh/static/papers/popl12-nested.pdf}",
  note = "[Online; accessed 10-November-2015]"
}

@misc{nst,
  title = {{Nomitive and Structural Typing}},
  howpublished = "\url{http://c2.com/cgi/wiki?NominativeAndStructuralTyping}",
  note = "[Online; accessed 13-November-2015]"
}

@misc{dynamic,
  author = {Jonathan Eifrig, Scott Smith, Valery Trifonov},
  title = {{Type Inference for Recursively Constrained Types and its Application to OOP}},
  howpublished = "\url{http://citeseerx.ist.psu.edu/viewdoc/download;jsessionid=F4F3BAF0FC6AB2A0681AA4428CFD4C4A?doi=10.1.1.92.6519&rep=rep1&type=pdf}",
  note = "[Online; accessed 13-November-2015]"
}
\end{filecontents}


\begin{document}
\nocite{*}
\maketitle
\hrule

% \raggedcolumns
\begin{multicols}{2}
% \raggedcolumns

\begin{abstract}
    Dynamic typing is a very hot field right now. It decreases the amount
    of work needed to complete simple tasks. Without doubt, dynamic languages
    have flexibility that is unmatched by any statically typed language no
    matter how powerful the type system. This flexibility comes at a cost, however.
    Namely in static analysis, particularly type verifiers. The lack of a static
    type system can lead to very onerous errors at runtime. This is where DuckTest
    comes in. DuckTest is a Python static analyzer that uses Data Flow Analysis
    and Type inference to detect TypeErrors statically while retaining as
    much flexibility as possible.
\end{abstract}

\section*{Introduction}

The duck test refers to the ever so popular statement ``If it looks like a duck,
swims like a duck, and quacks like a duck, then it probably is a duck.'' While
no sound logic for formal proofs, this type of logic has found its way into some
of the most popular programming languages today. These ``Duck'' type systems allow
an object $o_1$ to act as some type $t_1$ if $o_1$ implements all the features of
that type regardless of what the named type of $o_1$ actually is.

This type system makes is very flexible and allows the programmer to use
different types under the same context. Contrary to what many programmers
think, a duck type system is not synonomous with a \emph{dynamic} type
system, but the two are often together.

DuckTest is an application employing techniques such as Data Flow Analysis and
Type Inference to statically screen Python source code for potential type errors.
Currently DuckTest only supports a relatively limited subset of Python, but
extensions and improvements will continue to be added. DuckTest operates on Python,
but it itself is written in Haskell, rather than Python. This is a huge advantage
of DuckTest over other ``lint'' checkers because Haskell is very well suited for
this static program analysis and is also free of type errors at runtime.


\section*{Motivation}

Python and other dynamically typed languages are very useful tools to solve a
wide range of different problems. While these languages are very flexible and
very quick to prototype in, their very dynamic style makes it very difficult
for static analyzers to properly check the source for what one would consider
obvious errors.

Errors in dynamically typed languages can be notoriously onerous. This is
because a programmer may introduce a type error well before that type error
actually gets raised! Since these type errors are raised one at a time, to
find them you must run the program, potentially until the very end to find
type errors and debug them. For many programmers, this is simply out of the
question. This relentless and monotonous loop of run, crash, debug, fix is
enough to drive any programmer up the wall.

Yet there is a discrepancy here. Type errors most of the time happen from
a mistake by the programmer where an object was erroneously constructed (is
x supposed to be a list of tuples or a tuple of lists?) or where the programmer
accidentally leaves a function unimplemented, etc. The fact is, programmers
still tend to write dynamic code as they would static code, just without the
type annotations. If programmers can keep track of the attributes and avoid
writing erroneous code this way, why can't a machine do that and save us all
the time?

The answer is ``it's complicated.'' While indeed there are elements of Python
that are easy to check, what becomes insidious is the dynamism of the language
itself, particularly, the ability to check and change virtually anything at
runtime, for this reason, DuckTest has implemented one major restriction;
in order to reduce the number of false positives, the programmer must program
in a way that is consistent with \emph{Type Erasure Semantics}. Type Erasure
Semantics is a programming style that does not rely on type information at
runtime. In the context of Python, this means no dynamically changing the contents
of a class.

Even though DuckTest does recognize conditional branching based on the \emph{hasattr}
function and class equality and can adjust accordingly, it is highly recommended
not to this as it can confuse type checkers.

\section*{The Algorithm}

In order for type inference and checking to work in Python, it is paramount
to create a model for Python types that will allow the algorithm to reason
about types in a structured and consistent matter.

\subsection*{The Type Model}

There is a type system called \emph{Structural Typing}. A Structural
Type System is a system that uses an object's attributes as the identifier
of its type contrary to the majority of statically typed languages that
use some other characteristic like the name of the type.

A Structural Type model is an appealing way to model the types in Python
since in a structural type system, an object $o_1$ is compatible with
an object $o_2$ if $o_1$ implements all the attributes of the object $o_2$.
This attribute is very similar to how duck typing operates.

Structural Typing alone will not work as a model because it is too restrictive
in terms of lacking ways to model recursive types and function types; however,
the benefits are clear, which is why DuckTest uses a similar type model,
but with some changes.

In the source for DuckTest, there is a data structure used to represent
the types in python called \emph{PyType}. The PyType is a simple Algebraic
Data Type that follows the signature:

\begin{minted}{haskell}
data PyType = Scalar
                (Maybe String)
                (Map String PyType)
            | Functional
                [PyType]
                PyType
            | Alpha
                String
                PyType
            | Any
            | Void
\end{minted}

\subsubsection*{Scalar Type}

\mintinline{haskell}{Scalar (Maybe String) (Map String PyType)}

Under this model, scalars are the most common type. They alone are
the model of structural types. They contain an optional type name
(for debugging and error messages) and a mapping between attribute
names and associated types.

\subsubsection*{Functional Types}

\mintinline{haskell}{Functional [PyType] PyType}

A functional type is, quite obviously, a type representing functions.
Since python doesn't support currying, the functional type is implemented
as a basic procedural function type, that takes some arguments and returns
some other type. Python has the concept of varadic arguments, currently,
these are not modeled along with keyword and positional arguments.

\subsubsection*{$\alpha$-Types}

\mintinline{haskell}{Alpha String PyType}

The alpha type is an interesting concept borne out of trial and error. The
alpha type is used for recursive types. It is Essentially a marker so
the checking algorithm terminates properly. Consider the example of
a LinkedList. Under a purely structural model, this type may not be represented
since the type inferencer would never terminate as the type
would look something like

\maths
{LinkedList =\\\{data\hastype Any,\\next\hastype \{data\hastype Any,\\next\hastype \{data\hastype Any,\\next\hastype \{\ldots\}\}\}\}}
This is where the alpha type comes in. Under this system
a LinkedList would have the type

\maths{
LinkedList = \{data\hastype Any,\\next\hastype \alpha\ LinkedList\}
}

Alpha types exist purely as a pragmatic guard. The formal type algebra does
not need them, but that is simply because the formal algebra does not need
to concern itself with termination.

\subsubsection*{Any}

The Any type is an important concept. It has a dual role.  It behaves like the
infinite scalar. That is, the scalar with all possible attributes all of which
have the Any type themselves. This means the Any type is structurally compatible
with everything; however, the second role is that of the empty scalar, making
everything compatible as it. Essentially, the Any type is a wild card; it matches
everything and everything matches it.

\subsubsection*{Void}

The Void type, while important, is actually redundant. It is exactly equal to the
empty scalar. That is, the scalar with no attributes. Everything is compatible
as it, but it is only compatible as itself. Under strict rules, this would be the
type of \emph{None} since it cannot act as anything without crashing. However,
since \emph{None} is entrenched as an integral part of Python, it semantically
has the Any type.

Void is still very important as the type returned from functions lacking a return
type, as well as being effectively a zero to the algebra.


\subsection*{The Algebra}

I mentioned a couple of times in the previous section about an algebra for these types.
The algebra is an extremely important semantic to dealing with tracing the
types of variables throughout the program.

There are three primary operators from which the rest of the algebra is formed;
\emph{union, intersection, and difference}.

These three operators under this model are the basic tools used to infer and check
the types.

\subsubsection*{Union}

The union operator in the PyType algebra is the operator that takes
two types $t_1$ and $t_2$ and returns a type $t_3$ such that $t_3$ can
act as either $t_1$ or $t_2$.

There are some laws for unions, for example, \emph{Any} union with any
other type is \emph{Any}

\maths{Any\cup t = Any}

likewise, \emph{Void} union with any other type $t$ is $t$.

\maths{Void\cup t = t}

finally, the union operator is commutative

\maths{t_1\cup t_2 = t_2\cup t_1}

The key to the union type is that the type returned can itself match more
types, and can be used in more locations, but, it is also more difficult
to match against that type. This makes sense since $t_3$ is guaranteed
to have at least the number of attributes as its constituents.

The definition of union for scalars is fairly straightforward, it simply
recursively unions all the attributes of $t_1$ and $t_2$. So

\maths{%
    \begin{array}{ll}
        t_1 &= \{a \hastype  A, b \hastype  B_1, c \hastype  C_1\} \\
        t_2 &= \{c \hastype  C_2, b \hastype  B_2, d \hastype  D\} \\
        t_1 \cup_r t_2 &= \{a \hastype  A, b \hastype  B_1 \cup_r B_2, c \hastype  C_1 \cup_r C_2, d \hastype  D\} \\
    \end{array}
}

The union of a functional type and a scalar type results in a scalar with all
the same attributes of the scalar, but with an attribute \emph{\_\_call\_\_}
that is set to the functional type. This is because Python allows any type to
be called as long as that type implements \emph{\_\_call\_\_}.

\maths{%
    S \cup F = S\cup singleton\ ``\_\_call\_\_"\ F
}

Between two functions, the union type is a little counter-intuitive. Rather
than the union of all parameter types and the return types, it is actually the
\emph{intersection} of the parameter types and the union of the return types.

\maths{F\ p_1\ r_1\cup F\ p_2\ r_2\allowbreak=
    F\ \{p\cap q | (p,q) \in zip(p_1,p_2)\}\ (r_1\cup r_2)
}

The reasoning for this is because of what union is tring to portray. Union
is taking two types and returning a type that can match against either of
the operands.  By converse, an intersection of two types returns a type
that both operands can match against. So, qualitatively, if there
exists two functions $f_1$ and $f_2$ such that $f_1$ takes a $str$
as a parameter and $f_2$ takes an $int$, the union $f_1\cup f_2$ must
return a function that can act as either, meaning that it must accept
both an int and a string as an argument. This requires the parameter type to
be the \emph{intersection} between $int$ and $str$.

\subsubsection*{Intersection}

The last section gave a good dose of intersection, but we have not actually
defined what intersection is. In some ways, intersection is the convers of
union; while the union of $t_1$ and $t_2$ creates $t_3$ such that $t_3$ will
match with either $t_1$ or $t_2$, intersection is the opposite, it returns
a $t_3$ such that $t_1$ and $t_2$ both match against $t_3$. In a sense, union
always returs a type ``greater than or equal to'' the operands while
intersection always returns a type ``less than or equal to'' the operands.

Many of the rules are simply complements of those for union, although the two
operators are not necessarily opposites.

\maths{Void\cap t=Void}

\maths{Any\cap t=t}

\maths{t_1\cap t_2=t_2\cap t_1}

\maths{t\cap t = t\cup t = t}

The intersection of scalar types is very similar to the union of scalar types;
it takes the intersection of the two attribute maps and recursively
intersects the collisions.

\maths{\{a\hastype t_1, b\hastype t_2\}\cap \{a\hastype t_3, c\hastype t_4\} = \{a\hastype t_1\cap t_3\}}

Like union, the intersection between a scalar $s$ and a function $f$
is a scalar that is the intersection between $s$ and a promoted function%
\footnote{$promotion(f) = singleton\ ``\_\_call\_\_"\ f$}.

\maths{s\cap f=s\cap singleton\ ``\_\_call\_\_"\ f}

Finally, the intersection of two functions is the opposite of the union
between two functions, rather than taking the intersection between the
parameters, we actually take the union.

\maths{F\ p_1\ r_1\cap F\ p_2\ r_2 = F \{p\cup q| (p,q)\in zip(p_1,p_2)\} (r_1\cap r_2)}

\subsubsection*{Difference}

The final primary operator of this algebra is the difference operator.
The difference operator is in a way a third wheel to the other two. It
does not return a type that is particularly useful as a type. The difference
between a type $t_1\not = Any$ and $t_2\not = Void$ would be a type $t_3$ that cannont act
as either operand and such that either operand cannot act as $t_3$. Where
this operator does come in handy is determining if $t_1$ is larger than $t_2$.
If $t_2\geq t_1$ then $t_1\setminus t_2 = Void$, and this is important for
type matching. First some rules.

\maths{%
    \begin{array}{ll}
        Any\setminus t &= Any \\
        Void\setminus t &= Void \\
        t\setminus Void &= t \\
        t\setminus Any &= Void \\
        (t_1\cup t_2)\setminus t_1 = t_2
    \end{array}
}

Beyond that, the difference for two scalars is as one might guess,
the difference in the attribute maps using difference as a combining
function. So

\maths{
\{a\hastype A_1\}\setminus\{a\hastype A_2, b\hastype \beta\} =\\
{let\ d = A_1 \setminus A_2}\ in\\
{\quad if\ d = Void}\\{\qquad then\ \{\}}\\{\qquad else\ \{a\hastype d\}}
}

The difference between a scalar $s$ and a function $f$ is equal
to $s\setminus promoted(f)$ just like the other operators.

The real difference comes with functions. Unlike the previous,
the difference between two functions is actually the difference
between their parameters and the difference between their return
types:

\maths{F\ p_1\ r_1\setminus F\ p_2\ r_2 = F \{p\setminus q| (p,q)\in zip(p_1,p_2)\} (r_1\setminus r_2)}

With this equation above, it is impossible to actually, technically,
create $Void$ from this, so we introduce the concept of a ``Void Function''.
This type is a function type whose parameters all have type $Void$ and returns
a type $Void$ -- in a pure language a function with this type would be meaningless,
as it is saying it doesn't use the arguments and returns nothing. However,
Python is far from a pure language, so this type (specifically with 0 parameters) shows
up all the time.

\subsubsection*{Algebra on $\alpha$}

During this time, I have ignored the operators effect on the $\alpha$-type.
I did not forget about it, but intentionally left it out since it only exists
in a pragmatic sense. Practically, the semantics of the $\alpha$-type is to
``infect'' all types that touch it. So for any operator $\Box$,

\maths{%
\begin{array}{ll}
    \alpha\ t_1 \Box \alpha\ t_2 &= \alpha\ (t_1 \Box t_2)\\
    \alpha\ t_1 \Box t_2 &= \alpha\ (t_1 \Box t_2)\\
    t_1 \Box \alpha\ t_2 &= \alpha\ (t_1 \Box t_2)
\end{array}
}

While this look infinitely recursive and not actually practical,
these are just the semantics and since Haskell is lazy-evaluated
these values are not actually computed until the alpha value is
unwrapped.

\subsection*{Using the Algebra}

DuckTest uses three primary algorithms to accomplish its goal;
\emph{Type Inference by Observance}, \emph{Type Inference by Deduction}
and \emph{Type Matching}.

\subsubsection*{Type Inference by Observance}

\emph{Type Inference by Observance} is the art of inferring the
type by studying how it is used. This is the algorithm DuckTest
uses to infer the types of a function's parameters. It infers the
types by observing how those parameters are used in a function's
body.

The way this algorithm is implemented is quite simple. For each
parameter $p$ in the function's parameter list, if we observe in the
function's body an expression of the form $p.x$ the algorithm collects
this ``evidence'' and assigns a type to it; usually this is a singleton
type, in this case $p\hastype \{x\hastype \tau\}$. We call this an ``evidence type''; it is
not the full type of $p$, but it is a part of it. There's a problem
though, what is $\tau$? We obtain $\tau$ by running the same type observance
algorithm under the same body, but on the expression $p.x$ rather than $p$ itself.

Now, if we observe $p$ being passed to a function $f\hastype t_f$ as the
$n^{th}$ parameter, the evidence type is returned is the type of that
$n^{th}$ parameter in $t_f$. If for some reason there is an error (like
there is no $n^{th}$ parameter or the function $f$ cannot be resolved),
then the evidence type returned is $Void$.

If we observe $p$ being called, as in $p(arg0,\ldots)$ then we rerun
the algorithm on all the arguments to get the type of the arguments
$\tau_{a1}, \tau_{a2}, \ldots$ finally, we rerun the algorithm on
the whole expression to get the type of the return type $\tau_r$.
Finally, our evidence type is $Functional\ \{\tau_a,\ldots\}\ \tau_r$.

Lastly, if we observe $x = p$ then we rerun the algorithm on the expression
$x$ to get our new evidence type.

After the algorithm collects all the evidence types, it takes the union
of all of them to get the \emph{smallest complete type of $p$} which is
the observed to be the type of that parameter. During this phase of
the checking, the return type of this current function is said to be $Any$.
Return types are found using \emph{Type Inference by Deduction}.

\subsubsection*{Type Inference by Deduction}

\emph{Type Inference by Deduction} is a fairly simple algorithm. It primarily
consists of inferring the type of an object at its assignment based purely
on the return type of the function is is being set equal to. There is a little
more to the story than that, for example, type inference by deduction will
track the type through a branching statement. If in one branch, the variable
$v$ is deduced to have some type $t_1$ and in the other branch, $v\hastype t_2$,
then on the other side of the branches, $v\hastype t_1\cap t_2$.

This is how a return type from a function is deduced as well. The tracer algorithm
will go through a function body and find all the return expressions and
deduce the types returned. At the end, the function is said to return the
\emph{intersection} of all these types. This has the very nice side effect
of if a function is missing a return down a branch, the function is then
deduced to return $Void$ which will show up as a warning when a result
from the erroneous function is used.

\subsubsection*{Type Matching}

The last primary algorithm used is this concept of \emph{Type Matching}.
Type matching happens on arguments being passed to functions. When
an expression of the form $f(\ldots, arg, \ldots)$ is found, we match
$arg$ with the parameter type of $f$ at that position.

How does one tell if two types are match? The easiest way is to check
and see if the expected type $t_e$ (type of the parameter) differenced
with the actual type $t_a$ (deduced type of the argument) equals $Void$
($t_e\setminus t_a = Void$). If this statement is true, then the types
match\ldots in as strict sense. There is one exception to the rule.

Because of the dynamism of Python, and in the interest of keeping
the false positives to a minimum, the $Any$ type should match anything
and anything should match the $Any$ type. So the actual predicate for
match is defined as follows:

\maths{%
\begin{array}{ll}
    *\ matches\ Any\\
    Any\ matches\ *\\
  t_e\setminus t_a = Void &\Rightarrow  t_a\ matches\ t_e
\end{array}
}

It is very important to note that $t_a\ matches\ t_e \not\Rightarrow t_e\ matches\ t_a$
in fact, it's the near opposite. $t_a\ matches\ t_e \land t_a\not=t_e \Rightarrow \lnot(t_e\ matches\ t_a)$
In short, if $t_a$ matches $t_e$, then, unless $t_a = t_e$, $t_e$ will \emph{not} match $t_a$.

\subsection*{Tracing}

\subsubsection*{Assignments}
As alluded to, DuckTest uses a tracing algorithm that traces the possible
execution state of the interpreter with regards to types. The tracer essentially
is a giant fold over the parsed syntax tree using a glorified map of variables
to types as the state. In its simplest form, it looks for assignments
to variables and when it finds one it updates its map with the deduced
type for that variable.


\subsubsection*{Functions}
When a function definition is encountered, the function type is observed
and the tracer will then trace the function under a new state, the union
of the current state and the parameters of the function. When the tracer
returns from the function, it will have deduced the return type of that function
so the tracer can then continue under its old state except with the function name
now mapped to the deduced function type.

In actuality, when the tracer finds a function, it will actually defer
the trace of that function until it is actually used. The reason for this
is due to how scoping in Python works. In Python, a function can access
any variable from the outer closure even if that variable had not been assigned
to or otherwise declared before the point in the code that function definition
appeared. With out this deferring, false positives about undefined variables
from the outer scope would happen. This is especially true for mutually recursive
functions.

\subsubsection*{Classes}

The final element in tracing is how classes are handled. Classes in Python
have strange semantics, the full usage of will lead to many false positives
in DuckTest. This makes classes very challenging to deduce the type of, especially
taking into account potential recursion in the type.

Before beginning, it is important to note that with each class, there are in fact two types
we deal with: a static type and an instance type. A static type consists of
the functions and attributes of the class itself, namely the unbound functions
(functions not tied to an object) and static attributes. The instance type is
the type of the object created when constructing this class. The instance type
is the type that colloquially has the same name as the class name.

At a high-level, when the tracer finds a class, it
will first look for static assignments in that class. It will then
collect those into a type and assign that type to the class name.
It will also add a ``\_\_call\_\_'' attribute that returns type $\alpha Void$.

Then it will go through the class and deduce the types of all the unbound
functions in the class and union those with the type of that class.

Then we change the static type's ``\_\_call\_\_'' attribute to have the same type
as the ``\_\_init\_\_'' call, except with $\alpha\ Void$ as the return type.

The tracer will then create a \emph{instance} type from the class type.  It
does this first by collecting all the unbound methods and stripping the first
argument from all of them. Then it will collect all self-assignments
(expressions of patter $self.attribute = expr$) from the body of the class and
use those to determine what attributes a class has and deduce each attribute's
type.

Finally, there is one more step. Remember that $\alpha$ thing? That is for
recursive types like that linked list example. Before we are done constructing
the types for this class, we must go through and "rewire the alphas". This
means all attributes and subattributes of the instance type and static types
with the type $\alpha\ Void$, we change to be $\alpha\ t$ where $t$ is our bound
type.

At the end of it all, we have created a new state with one measly change,
the class name has been mapped to the static type of the class. That's counter-intuitive,
but remember that the instance type still is reachable as the return type
of the static type's ``\_\_call\_\_'' method.

\subsubsection*{Imports}

Import statements are actually a refreshingly simple task after classes.

One may have already noticed that the state the tracer carries around is
already actually a scalar type! Think about it; it has a map of variable names
to types, which is exactly what a structural type has. This makes it very easy
to deal with imports. Import statements in Python (generally) refer to some file
on disk. Finding this file is a relatively simple (although tedious) process, but
once the file is located, we can simply trace the file like any other, get the
finishing state of the tracer, lift that state to a type, and set the appropriate
variable to that type.

There is one wrinkle with this, and that is that modules can import each other
recursively. To combat this, DuckTest will cache the ``type'' of a module as
$Any$ before tracing that module. When the trace is complete, DuckTest will
then cache the real type of the module so as to quickly be able to access
it again if an import appears elsewhere.

\section*{Results}

DuckTest is still in early stages of development and is still far from being
useful on large-scale applications. In spite of this, however, DuckTest
is showing great promise and here are some examples to show.

All of these examples were also tested against frosted, pep257, pep8,
prospector, pyflakes and pylint for good measure.

\subsection*{Simple Duck Test}
\inputcode{examples/duck.py}

Under this program, no lint checker found the error
on line 29 of the program. DuckTest was able to find
this error. It was able to deduce that the argument
to \emph{doStuff} needs to have a \emph{swim} method,
which does not exist for a \emph{Person} class. The actual
error returned was
\begin{Verbatim}[commandchars=&\[\]]
&fvtextcolor[red][writeup/examples/duck.py(29:1)]:
In expression `person`:
    '&fvtextcolor[darkgreen][Person]' incompatible as
        '&fvtextcolor[darkgreen][{ swim, talk, walk }]'.
Missing attributes needed:
    '&fvtextcolor[darkgreen][swim :: () -> Void]'
\end{Verbatim}
This is fairly easy to read. It gives as well as a list of
needed parameters.

\subsection*{Recursion and String Concatenation}

\inputcode{examples/collatz.py}

This program has a couple of interesting factors in it.
There is recursion, but the real error is trying to concatenate
an int with a string. This is personally one of the most annoying
runtime errors of them all, but DuckTest can detect these bugs.


\begin{Verbatim}[commandchars=&\[\]]
&fvtextcolor[red][writeup/examples/collatz.py(9:7)]:
    In expression `collatz(int(9))`:
     '&fvtextcolor[darkgreen][int]' incompatible as '&fvtextcolor[darkgreen][str]'.
      Missing attributes needed: ...
\end{Verbatim}

{\footnotesize{(The \ldots are in lieu of the massive number of attributes strings have that ints do not)}}

\subsection*{Linked Lists and Stuff}

\inputcode{examples/linkedlist.py}

This is an example of a recursive type. It is actually a complex problem
to infer the types in this program, yet DuckTest is able to trace this and
find a couple of errors.

\begin{Verbatim}[commandchars=&\[\]]
&fvtextcolor[red][writeup/examples/linkedlist.py(20:7)]:
 The subexpression head.next.next
  may have no attribute
   'q' (head.next.next :: alpha)

&fvtextcolor[red][writeup/examples/linkedlist.py(18:1)]:
 In expression `cursor`:'&fvtextcolor[darkgreen][LinkedList]'
  incompatible as '&fvtextcolor[darkgreen][{ __add__ }]'.
   Missing attributes needed:
    '&fvtextcolor[darkgreen][__add__ :: ( :: Void) -> Void]'
\end{Verbatim}

DuckTest successfully found the error where 2 was attempted to be added to
a \emph{LinkedList} as well as an error where \emph{LinkedList} does not have
the attribute \emph{q} (which is pretty obvious).

\section*{What Next}

As you can see, DuckTest has shown some promise. It is currently able
to detect pretty obvious type errors. Areas for expansion include
having DuckTest figure out the general types of lists and other
containers to make sure the types being inserted and removed are
consistent with what are expected.

Of course this sort of feature would have to come under the guard of
some command-line option because the user could always be intending
to create an ``object list'' that just happens to be populated with
all strings.

To take this to more of an extreme, there are a number of ``paranoid'' warnings
DuckTest could be modified to emit if the user wishes. Potential topics are
detecting if a type could be $None$ and if that variable is used without a
guard.

Ultimately, in order for DuckTest to be useful, it needs a vastly expanded
model for all of the many builtins Python supports as well as better
handling for imports. (The ``professional'' code in the standard library
is extremely dynamic!)

Support for things like tuples, tuple unpacking, varadic and keyword arguments
are yet still unsupported. Although unsupported, it is not impossible and
models for their handling already do exist.

This, however, is a stepping stone to greater things. When first starting out
on the project, I was not sure if this would be possible at all thinking Python
was simply too fluid to do anything; however, with a solid type model and inference
engine it really is possible to do at least a good amount of static checking
on Python source code.

\bibliographystyle{abbrv}
\bibliography{bib}

\end{multicols}
\end{document}
